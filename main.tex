\documentclass[12pt]{article}
	\usepackage[T1]{fontenc}
	\usepackage[utf8]{inputenc}
	\usepackage[british]{babel}
	\usepackage[a4paper]{geometry}
	\geometry{verbose,tmargin=3cm,bmargin=3.5cm,lmargin=4cm,rmargin=3cm,marginparwidth=70pt}
	\setcounter{secnumdepth}{3}
	\setcounter{tocdepth}{3}
	\usepackage{prettyref}
	\usepackage{textcomp}
	\usepackage{setspace}
	\usepackage{indentfirst}
	\usepackage{fancyhdr}
	\usepackage{url}
	\usepackage[normalem]{ulem}
	\usepackage[table, fixpdftex]{xcolor}
	\usepackage{algpseudocode}
	\usepackage{bigstrut}
	\usepackage{enumitem}

	% package hyperref
	\usepackage{hyperref}

	% biblatex
	\usepackage[style=authoryear-icomp,natbib=true,maxcitenames=2, maxbibnames=11,backend=biber,pagetracker=page,hyperref=true]{biblatex} \usepackage{csquotes}
	\renewcommand*{\bibsetup}{%
		\interlinepenalty=10000\relax % default is 5000
		\widowpenalty=10000\relax
		\clubpenalty=10000\relax
		\raggedbottom
		\frenchspacing
		\biburlsetup}
	% fixes the page number of the first page of each chapter
	\fancypagestyle{plain}{
			\fancyhead{}
			\renewcommand{\headrulewidth}{0pt}
			\renewcommand{\footrulewidth}{0pt}
			\fancyfoot[OC]{\begin{flushright}\thepage\end{flushright}}
	}
	
	% fancy headers for the thesis
	\fancyhead{}
	\fancyhead[RO]{\slshape \nouppercase \rightmark}
	\fancyfoot[OC]{\begin{flushright}\thepage\end{flushright}}
	\renewcommand{\headrulewidth}{0.4pt}
	\setlength{\headheight}{14pt}
	
	
	% add bibliography database
	\addbibresource{BA Kopie.bib}
	
	% space between biblio items
	\setlength\bibitemsep{1.7\itemsep} 
	
	% title without ""
	\DeclareFieldFormat[inbook]{title}{#1}
	% non-italic
	\DeclareFieldFormat[online]{tlaitle}{#1} 
	% title unquoted
	\DeclareFieldFormat[article]{title}{#1} 
	% no pp. 
	\DeclareFieldFormat[article]{pages}{#1} 
	% bold volume
	\DeclareFieldFormat*{volume}{\mkbibbold{#1}\setpunctfont{\textbf}}
	
	% no in:
	\renewbibmacro{in:}{} 
	
	% (volume)
	\renewbibmacro*{volume+number+eid}{%
			\printfield{volume}%
			%\setunit*{\adddot}% DELETED
			% \setunit*{\addnbspace}% NEW (optional); there's also \addnbthinspace
			\printfield{number}%
			% \setunit{\addcomma\space}%
			\printfield{eid}}
	\DeclareFieldFormat[article]{number}{\mkbibparens{#1}} 
	
	% edition.
	\DeclareFieldFormat{edition}%
	{(\ifinteger{#1}%
			{\mkbibordedition{#1}\addthinspace{}ed.}%
			{#1\isdot}).}
	
	% publisher and location position
	\renewbibmacro*{publisher+location+date}{%
			\printlist{publisher}%
			\setunit*{\addcomma\space}%
			\printlist{location}%
			\setunit*{\addcomma\space}%
			\usebibmacro{date}%
			\newunit}
	
	% shortauthor before author
	\renewbibmacro*{begentry}{%
			\ifkeyword{Key}{\sffamily}{}%
			\iffieldundef{shorthand}
			{}
			{\global\undef\bbx@lasthash
					\printfield{shorthand}%
					\addcolon\space}%
			\ifboolexpr{test {\usebibmacro{bbx:dashcheck}} or test {\ifnameundef{shortauthor}}}%
			{}%
			{\printnames{shortauthor}%
					\addspace\textendash\space}}

\title{Proposal Bachelor Thesis}
\author{Leopold Ingenohl}

\begin{document}
\maketitle

\section{Introduction}
Much attention has been recently given to Schedule 13D Filings, the beneficial ownership form many investors must file to report their equity holdings, identifying them as active investors \citep{Giglia2018}. One of the reasons could be the imputed activity of the investor, namely representing an increased likelihood of the firm becoming a takeover target \citep{Brigida2012}. The markets' reflections to investor activism are positive and significant average abnormal returns around the filing date, as \citet{Brav2008} have shown.

In particular, \citet{Akhigbe2007} observed that stakes acquired by corporate investors are more likely to result in a full acquisition of the target. By applying \citep{Brav2008} findings, filings from corporate investors should therefore lead to higher abnormal returns. In fact, \citet{Brigida2012} claim that non-financial corporations have a runup (the abnormal return around the filing date) almost 13 percent higher compared to financial corporations.

The strong relation between schedule 13D filings of non-financial corporations and the respective market reaction motivate this paper.


\section{Objective of the Paper}
The main objective of the paper is to determine how important - to participants on the stock exchange - the financial condition (strength) of the engaging activist investor is. In particular, the financial structure of non-financial corporations that have filed a Schedule 13D \citep{Bell2017}. 
For the analysis of this relation, key figures computed from the last filed report of the investor will be paired with the respective abnormal returns around the 13D filing date. (I could not find literature on parameters determing a company's financial strength/condition yet. A possible application could be Piotroskis' f-score \citep{Piotroski2000}.)\\*
The hypothesis could be stated as the following: The higher the interdependence between the financial structure of the investor and the abnormal returns, the more important is the investors' financial power to other participants on the market.
\pagebreak

In a second step, partial results of the previous analysis could be used to compare the financial structure of the filer (investor) with that of the target. The aim of this section would be to identify possible effects of different "financial investor/target ratios" - e.g. do corporate investors only target poorly performing firms \citep{Klein2009}? (to be discussed!)

\section{Procedure of Analysis}

\begin{enumerate}
\item Generate a representative sample of SC 13D Filings based on the following restrictions / qualifications
	
	\begin{enumerate}
	\item Both, subject and filer, have to be non-financial corporations (characteristics/filter will soon be defined, such as 10K-Filings). For instance, no private investment firms and an exclusion of filings from non-financial and utility industries \citep{Brigida2012}.
 	\item Both, subject and filer, have to be listed in the COMPUSTAT and CRSP data bases.
	\item The sample is restricted to filings from 1995 up to ?.
	\item to be discussed
	\end{enumerate}
	
\item Gather the COMPUSTAT data corresponding to the corporations of the previously filtered sample of 13D Filings.
\item Allocate key financials, representing the financial condition, to both, subject and filer.
\item Gather the corresponding stock date-data via CRSP.
\item Estimation of the abnormal returns, based on the previously compiled stock data through an event study \citep{ang2011} \citep{Fama1992} \citep{Kolari2010}. 
\item Merging of investors' key financials with CRSP data.

\end{enumerate}

\pagebreak
\printbibliography[title=Relevant Literature (so far)]
\end{document}
