\documentclass[11pt,openright]{article}
	\usepackage[T1]{fontenc}
	\usepackage[latin9]{inputenc}
	\usepackage[british]{babel}
	\usepackage[a4paper]{geometry}
	\geometry{verbose,tmargin=3cm,bmargin=3.5cm,lmargin=4cm,rmargin=3cm,marginparwidth=70pt}
	\setcounter{secnumdepth}{3}
	\setcounter{tocdepth}{3}
	\usepackage{prettyref}
	\usepackage{textcomp}
	\usepackage{setspace}
	\usepackage{indentfirst}
	\usepackage{fancyhdr}
	\usepackage{url}
	\usepackage[normalem]{ulem}
	\usepackage[table, fixpdftex]{xcolor}
	\usepackage{algpseudocode}
	\usepackage{bigstrut}
	\usepackage{enumitem}

	% biblatex
	\usepackage[style=authoryear-icomp, natbib=true, maxcitenames=2, maxbibnames=11, backend=biber, hyperref=true, pagetracker=page]{biblatex} \usepackage{csquotes}
	\renewcommand*{\bibsetup}{%
		\interlinepenalty=10000\relax % default is 5000
		\widowpenalty=10000\relax
		\clubpenalty=10000\relax
		\raggedbottom
		\frenchspacing
		\biburlsetup}
	
	% package hyperref
	\usepackage[unicode=true,pdfusetitle,
	bookmarks=true,bookmarksnumbered=false,bookmarksopen=false,
	breaklinks=false,pdfborder={0 0 0.7},backref=false,colorlinks=false]
	{hyperref}

	% fixes the page number of the first page of each chapter
	\fancypagestyle{plain}{
			\fancyhead{}
			\renewcommand{\headrulewidth}{0pt}
			\renewcommand{\footrulewidth}{0pt}
			\fancyfoot[OC]{\begin{flushright}\thepage\end{flushright}}
	}
	
	% fancy headers for the thesis
	\fancyhead{}
	\fancyhead[LE]{\slshape \nouppercase \leftmark}
	\fancyhead[RO]{\slshape \nouppercase \rightmark}
	\fancyfoot[EC]{\begin{flushleft}\thepage\end{flushleft}}
	\fancyfoot[OC]{\begin{flushright}\thepage\end{flushright}}
	\renewcommand{\headrulewidth}{0.4pt}
	\setlength{\headheight}{14pt}
	
	
	% add bibliography database
	\addbibresource{BA Kopie.bib}
	
	% space between biblio items
	\setlength\bibitemsep{1.7\itemsep} 
	
	% title without ""
	\DeclareFieldFormat[inbook]{title}{#1}
	% non-italic
	\DeclareFieldFormat[online]{title}{#1} 
	% title unquoted
	\DeclareFieldFormat[article]{title}{#1} 
	% no pp. 
	\DeclareFieldFormat[article]{pages}{#1} 
	% bold volume
	\DeclareFieldFormat*{volume}{\mkbibbold{#1}\setpunctfont{\textbf}}
	
	% no in:
	\renewbibmacro{in:}{} 
	
	% (volume)
	\renewbibmacro*{volume+number+eid}{%
			\printfield{volume}%
			%\setunit*{\adddot}% DELETED
			% \setunit*{\addnbspace}% NEW (optional); there's also \addnbthinspace
			\printfield{number}%
			% \setunit{\addcomma\space}%
			\printfield{eid}}
	\DeclareFieldFormat[article]{number}{\mkbibparens{#1}} 
	
	% edition.
	\DeclareFieldFormat{edition}%
	{(\ifinteger{#1}%
			{\mkbibordedition{#1}\addthinspace{}ed.}%
			{#1\isdot}).}
	
	% publisher and location position
	\renewbibmacro*{publisher+location+date}{%
			\printlist{publisher}%
			\setunit*{\addcomma\space}%
			\printlist{location}%
			\setunit*{\addcomma\space}%
			\usebibmacro{date}%
			\newunit}
	
	% shortauthor before author
	\renewbibmacro*{begentry}{%
			\ifkeyword{Key}{\sffamily}{}%
			\iffieldundef{shorthand}
			{}
			{\global\undef\bbx@lasthash
					\printfield{shorthand}%
					\addcolon\space}%
			\ifboolexpr{test {\usebibmacro{bbx:dashcheck}} or test {\ifnameundef{shortauthor}}}%
			{}%
			{\printnames{shortauthor}%
					\addspace\textendash\space}}

\title{Proposal Bachelor Thesis}
\author{Leopold Ingenohl}

\begin{document}
\maketitle

\section{Introduction}
Much attention has been recently given to Schedule 13D Filings, the beneficial ownership form many investors must file to report their equity holdings \citep{Giglia2018} and identifying them as active investors. By filing the schedule, the investor represents an increased likelihood of the firm becoming a takeover target \citep{Brigida2012}. The market reflects this possibility with positive and significant average abnormal returns around the filing date(Brav et. al). \\
In particular, \citet{Akhigbe2007} have shown that stakes acquired by corporate bidders are more likely to result in a full acquisition of the target. By continuing the argumentation above, this should result in increased abnormal returns.  In fact, \citet{Brigida2012} came to the conclusion that non-financial corporations have a runup  almost 13 percent higher compared to financial corporations. The large effect of of non-financial corporations on the abnormal return motivates the analysis of this paper.


\section{Objective and Importance of the Paper}
The main objective of this paper is to determine how important,to participants on the stock exchange, the financial power of large, activist investors (non-financial) is. For analyzing this relation, key financials of the investor will be paired with the respective abnormal returns around the 13D filing date. In other words - the higher the interdependence between abnormal returns and financial structure of the investor, the more important is the investors' financial power of large investors to other participants on the stock exchange.
Secondly, partial results of the previous analysis shall be used to compare the key financials of the investor with those of the target. The aim is to identi fy possible effects of different "investor-target ratios" on the respective abnormal returns prior to the SC 13D Filing. 

\section{Value-Added}

\section{Relevant Literature}

\section{Procedure of Analysis}

\begin{enumerate}
\item Generate a representative sample of SC 13D Filings based on the following restrictions/qualifications

	\begin{enumerate}
	\item Both, subject and filer, have to be a corporation (characteristics/filter will soon be defined) and not an investment firm.
  \item Both, subject and filer, have to be listed in the COMPUSTAT and CRSP data bases.
	\item The sample is restricted to the timeline from 1995-2010.
	\item Exclusion of filings from non-financial and utility industries \citep{Brigida2012}
	\item to be discussed
	\end{enumerate}

\item Gather the COMPUSTAT data corresponding to the corporations in the previously filtered sample of 13D Filings.
\item Allocate Key Financials to both, subject and filer
\item Gather the stock data, corresponding to the date of the filings, via CRSP.
\item Based on the stock data compiled around the date of the filing, the abnormal returns are calculated. 
\item How important is the financial power of large investors? Analysis of abnormal returns compared to key financials of the corresponding firms

\end{enumerate}


\section{Problems}






\section{Data and Analysis}

\printbibliography
\end{document}
